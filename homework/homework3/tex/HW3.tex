\documentclass[10pt]{article}
\usepackage[UTF8]{ctex}

\usepackage[utf8]{inputenc} % allow utf-8 input
\usepackage{amsthm}
\usepackage{amsmath,amscd}
\usepackage{amssymb,array}
\usepackage{amsfonts,latexsym}
\usepackage{graphicx,subfig,wrapfig}
\usepackage{times}
\usepackage{psfrag,epsfig}
\usepackage{verbatim}
\usepackage{tabularx}
\usepackage[pagebackref=true,breaklinks=true,letterpaper=true,colorlinks,bookmarks=false]{hyperref}
\usepackage{cite}
\usepackage{algorithm}
\usepackage{multirow}
\usepackage{caption}
\usepackage{algorithmic}
%\usepackage[amsmath,thmmarks]{ntheorem}
\usepackage{listings}
\usepackage{color}
\usepackage{bm}



\newtheorem{thm}{Theorem}
\newtheorem{mydef}{Definition}

\DeclareMathOperator*{\rank}{rank}
\DeclareMathOperator*{\trace}{trace}
\DeclareMathOperator*{\acos}{acos}
\DeclareMathOperator*{\argmax}{argmax}


\renewcommand{\algorithmicrequire}{ \textbf{Input:}}
\renewcommand{\algorithmicensure}{ \textbf{Output:}}
\renewcommand{\mathbf}{\boldsymbol}
\newcommand{\mb}{\mathbf}
\newcommand{\matlab}[1]{\texttt{#1}}
\newcommand{\setname}[1]{\textsl{#1}}  
\newcommand{\Ce}{\mathbb{C}}
\newcommand{\Ee}{\mathbb{E}}
\newcommand{\Ne}{\mathbb{N}}
\newcommand{\Se}{\mathbb{S}}
\newcommand{\norm}[2]{\left\| #1 \right\|_{#2}}

\newenvironment{mfunction}[1]{
	\noindent
	\tabularx{\linewidth}{>{\ttfamily}rX}
	\hline
	\multicolumn{2}{l}{\textbf{Function \matlab{#1}}}\\
	\hline
}{\\\endtabularx}

\newcommand{\parameters}{\multicolumn{2}{l}{\textbf{Parameters}}\\}

\newcommand{\fdescription}[1]{\multicolumn{2}{p{0.96\linewidth}}{

		\textbf{Description}

		#1}\\\hline}

\newcommand{\retvalues}{\multicolumn{2}{l}{\textbf{Returned values}}\\}
\def\0{\boldsymbol{0}}
\def\b{\boldsymbol{b}}
\def\bmu{\boldsymbol{\mu}}
\def\e{\boldsymbol{e}}
\def\u{\boldsymbol{u}}
\def\x{\boldsymbol{x}}
\def\v{\boldsymbol{v}}
\def\w{\boldsymbol{w}}
\def\N{\boldsymbol{N}}
\def\X{\boldsymbol{X}}
\def\Y{\boldsymbol{Y}}
\def\A{\boldsymbol{A}}
\def\B{\boldsymbol{B}}
\def\y{\boldsymbol{y}}
\def\cX{\mathcal{X}}
\def\transpose{\top} % Vector and Matrix Transpose

%\long\def\answer#1{{\bf ANSWER:} #1}
\long\def\answer#1{}
\newcommand{\myhat}{\widehat}
\long\def\comment#1{}
\newcommand{\eg}{{e.g.,~}}
\newcommand{\ea}{{et al.~}}
\newcommand{\ie}{{i.e.,~}}

\newcommand{\db}{{\boldsymbol{d}}}
\renewcommand{\Re}{{\mathbb{R}}}
\newcommand{\Pe}{{\mathbb{P}}}
\newenvironment{problem}[2][Problem]{\begin{trivlist}
\item[\hskip \labelsep {\bfseries #1}\hskip \labelsep {\bfseries #2.}]}{\end{trivlist}}

\hyphenation{MATLAB}

\usepackage[margin=1in]{geometry}

\begin{document}

\title{	Numerical Optimization, 2023 Fall\\Homework 3}
\date{Due 23:59 (CST), Nov. 16, 2023 }

\author{
    Name: \textbf{Zhou Shouchen} \\
	Student ID: 2021533042
}

\maketitle
\newpage

%===============================

\begin{problem}{1}
    Prove the dual of the dual of a linear programming (standard form) is itself.\textcolor{red}{[25pts]}
\end{problem}
We can prove this with the help of the Duality Scheme.\\
Consider a linear programming that is in standard form:\\
\begin{equation}
\begin{aligned}
\min_{\bm{x}} \quad & \bm{c}^T\bm{x} \\
\text { s.t. } \quad & A\bm{x} = \bm{b} \\
& \bm{x} \geq \bm{0}
\end{aligned}
\end{equation}

The Lagrangian of the above linear programming is $L(\bm{x},\bm{\lambda}) = \bm{c}^T\bm{x} - \bm{\lambda}^T(A\bm{x} - \bm{b})$.\\

Since for the primal question, the variables are $\bm{x}\geq 0$, so for $x_i\geq 0$, the dual constrain is that:\\
$$a_i^T\bm{\lambda} \leq c_i$$
where $a_i$ is the $i$-th column of $A$.\\

So for the dual problem, the constrains are $A^T\bm{\lambda} \leq \bm{c}$.\\

And since for the primal question, the constrain it that $A\bm{x} = \bm{b}$, i.e.$\sum\limits_{j=1}^na_{ij}x_j=b_i$.\\
So for the dual question, the variables $\bm{\lambda}$ is free.\\

So the dual problem is that:\\
\begin{equation}
\begin{aligned}
\max_{\bm{\lambda}} \quad & \bm{\lambda}^T\bm{b} \\
\text { s.t. } \quad & A^T\bm{\lambda} \leq \bm{c} \\
& \bm{\lambda} \text{ is free}
\end{aligned}
\end{equation}

To easily get the dual of the dual problem, we can first convert the objective function of the dual problem to a minimization problem. And take $M=A^T$,
the dual problem becomes:
\begin{equation}
\begin{aligned}
\min_{\bm{\lambda}} \quad & \bm{\lambda}^T\bm{(-b)} \\
\text { s.t. } \quad & M\bm{\lambda} \leq \bm{c} \\
& \bm{\lambda} \text{ is free}
\end{aligned}
\end{equation}

The Lagrangian of the dual problem is:\\
$$L(\bm{\lambda},\bm{y}) = \bm{\lambda}^T\bm{(-b)} - \bm{y}^T(M\bm{\lambda} - \bm{c})$$


Since for the dual question, the variables are $\bm{\lambda}\text{ is free}$, so for $\lambda_i\text{ is free}$, the dual of the dual constrain is that:\\
$$m_i^T\bm{\lambda} = (-\bm{b})_i = -b_i$$
where $m_i$ is the $i$-th column of $M$.\\

So for the dual problem, the constrains are
$$M^T\bm{y} = \bm{-b}$$

And since for the dual question, the constrain it that $M\bm{\lambda} \leq \bm{c}$, i.e.$\sum\limits_{j=1}^nm_{ij}\lambda_j\leq c_i$.\\
so for the dual question, the variables are $\bm{y}\leq\bm{0}$.

So the dual of the dual problem is that:\\
\begin{equation}
\begin{aligned}
\max_{\bm{y}} \quad & \bm{c}^T\bm{y} \\
\text { s.t. } \quad & M^T\bm{y} = \bm{-b} \\
& \bm{y} \leq \bm{0}
\end{aligned}
\end{equation}

We can take $\bm{x}=-\bm{y}$. And since $M=A^T$, so $M^T=(A^T)^T=A$.

Consider the objective function is 
$$\max\limits_{\bm{y}} \quad \bm{c}^T\bm{y}$$
We can convert it into a minimization problem by taking $-\bm{y}$ as the variable.\\
i.e.
$$\min\limits_{\bm{y}} \quad \bm{c}^T\bm{(-y)}$$
And since we have $\bm{x}=-\bm{y}$, so we can convert the above minimization problem into:
$$\min\limits_{\bm{x}} \quad \bm{c}^T\bm{x}$$

And for the first constrain, we have:
$$M^T\bm{y} = \bm{-b}$$
Since $M^T=A$, and move the "-" from the right to the left, we have:
$$A\bm{(-y)} = \bm{b}$$
i.e.
$$A\bm{x} = \bm{b}$$

For the second constrain, we have:
$$\bm{y}\leq\bm{0}$$
We can convert it into:
$$\bm{-y}\geq\bm{0}$$
i.e.
$$\bm{x}\geq\bm{0}$$

So with the convertions above, we can get that the dual of the dual problem is that:
\begin{equation}
\begin{aligned}
\min_{\bm{x}} \quad & \bm{c}^T\bm{x} \\
\text { s.t. } \quad & A\bm{x} = \bm{b} \\
& \bm{x} \geq \bm{0}
\end{aligned}
\end{equation}

Which is exactly same with the primal problem.\\
So above all, the dual of the dual of a linear programming (standard form) is itself.\\

\newpage

% %===============================

\begin{problem}{2}
    Prove the dual objective increases after a pivot of the dual simplex method.\textcolor{red}{[25pts]}
\end{problem}

Consider the dual simplex method.\\
Suppose that the current state is feasible, and after the pivot, it is also feasible.\\
Then for the pivot process, we have:\\
$r^T\geq 0$ always satisfies, this is to make sure the dual problem is feasible.\\
As for choosing the pivot element, we choose the $p$-th row s.t. the current $b_p$ in the tableau is nagetive.\\
i.e. $b_p<0$.\\
Suppose that the $p$-th row is $y_{p1},y_{p2},\cdots,y_{pn},b_p$, and we choose the pivot element $a_{pq}$.\\
s.t. $\hat{\epsilon}=\dfrac{r_q}{-y_{pq}} = \min\{\dfrac{r_i}{-y_{pi}}\mid a_{pi}<0,i=1,\cdots,n\}$.\\
Then we pivot with $y_{pq}$, and we need to update the tableau to make the $r_q$ become $0$ by:\\
Let the last line of the simplex tableau adds $\hat{\epsilon}$ times the $p$-th row.\\
So we have
$$r_q'=r_q+\hat{\epsilon}y_{pq}=r_q+\dfrac{r_q}{-y_{pq}}y_{pq}=0$$
$$-f'=-f+\hat{\epsilon}b_p=-f+\dfrac{r_q}{-y_{pq}}b_p$$

We know that the lower-right corner if $-f$, where $f$ is the dual objective value of the dual problem.\\
So after the pivot, the lower-right corner becomes:\\
$$-f'= -f - \frac{r_q}{y_{pq}}b_p$$
Since $r_q>0,y_{pq}<0,b_p<0$, so $\frac{r_q}{y_{pq}}b_p>0$
So
$$-f= -f - \frac{r_q}{y_{pq}}b_p < -f'$$
i.e.
$$f'>f$$
So above all, the dual objective increases after a pivot of the dual simplex method.\\

\newpage

% %===============================

\begin{problem}{3}
    Let $L(\bm{x}, \bm{\lambda})$ be the Lagrangian of a linear programming problem, and $(\bm{x}^*, \bm{\lambda}^*)$ be the optimal primal-dual solution. Prove that $$L(\bm{x}, \bm{\lambda}^*) \geq L(\bm{x}^*, \bm{\lambda}^*) \geq L(\bm{x}^*, \bm{\lambda}),$$ for any primal feasible $\bm{x}$ and dual feasible $\bm{\lambda}$.\textcolor{red}{[25pts]}
\end{problem}

Suppose that the primal problem is that:
\begin{equation}
\begin{aligned}
\min_{\bm{x}} \quad & \bm{c}^T\bm{x} \\
\text { s.t. } \quad & A\bm{x} = \bm{b} \\
& \bm{x} \geq 0
\end{aligned}
\end{equation}

And the dual problem is that:
\begin{equation}
\begin{aligned}
\max_{\bm{\lambda}} \quad & \bm{\lambda}^T\bm{b} \\
\text { s.t. } \quad & A^T\bm{\lambda} \leq \bm{c} \\
\end{aligned}
\end{equation}

And the Lagrangian of the primal problem is that:
$$L(\bm{x},\bm{\lambda}) = \bm{c}^T\bm{x} + \bm{\lambda}^T(A\bm{x} - \bm{b})$$

Since we can considering the primal feasible $\bm{x}$ and dual feasible $\bm{\lambda}$, so we have:
$$A\bm{x} = \bm{b}$$
$$\bm{c}^T\bm{x} \geq \bm{c}^T\bm{x}^*$$
If we put $A\bm{x} - \bm{b} = \bm{0}$ into the Lagrangian, we have:
$$L(\bm{x},\bm{\lambda}) = \bm{c}^T\bm{x} + \bm{\lambda}^T(A\bm{x} - \bm{b}) = \bm{c}^T\bm{x} + \bm{\lambda}^T\bm{0} = \bm{c}^T\bm{x}$$
And since $\bm{c}^T\bm{x} \geq \bm{c}^T\bm{x}^*$, and $L(\bm{x},\bm{\lambda})$ do not contain $\bm{\lambda}$, so we have:
$$L(\bm{x},\bm{\lambda}) \geq L(\bm{x}^*,\bm{\lambda}) = L(\bm{x}^*,\bm{\lambda})$$

So above all, we have prove that 
$$L(\bm{x}, \bm{\lambda}^*) \geq L(\bm{x}^*, \bm{\lambda}^*) \geq L(\bm{x}^*, \bm{\lambda})$$

\newpage

% %===============================

\begin{problem}{4}
    Construct a linear programming problem for which both the primal and the dual problem has no feasible solution.\textcolor{red}{[25pts]}

Construct a linear programming problem that is:\\
\begin{equation}
\begin{aligned}
\min_{x_1,x_2} \quad & x_1-2x_2 \\
\text { s.t. } \quad & x_1 - x_2 \leq 1 \\
& x_1 - x_2 \geq 2 \\
& x_1,x_2 \leq 0
\end{aligned}
\end{equation}

Since it is impossible to satisfy $x_1 - x_2 \leq 1$ and $x_1 - x_2 \geq 2$ at the same time, so the primal problem has no feasible solution.\\
And the dual problem is that:\\
\begin{equation}
\begin{aligned}
\max_{\lambda_1,\lambda_2} \quad & \lambda_1 + 2\lambda_2 \\
\text { s.t. } \quad & \lambda_1 + \lambda_2 \leq 1 \\
& -\lambda_1 - \lambda_2 \leq -2 \\
& \lambda_1\leq 0,\lambda_2 \geq 0
\end{aligned}
\end{equation}
The second constrain $-\lambda_1 - \lambda_2 \leq -2$ can be written as $\lambda_1 + \lambda_2 \geq 2$.\\
Since it is impossible to satisfy $\lambda_1 + \lambda_2 \leq 1$ and $\lambda_1 + \lambda_2 \geq 2$ at the same time, so the dual problem has no feasible solution.\\

So above all, the above construction's primal and dual problem has no feasible solution.\\

\end{problem}

\end{document}


