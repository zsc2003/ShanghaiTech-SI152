\documentclass[10pt]{article}
\usepackage[UTF8]{ctex}

\usepackage[utf8]{inputenc} % allow utf-8 input
\usepackage{amsthm}
\usepackage{amsmath,amscd}
\usepackage{amssymb,array}
\usepackage{amsfonts,latexsym}
\usepackage{graphicx,subfig,wrapfig}
\usepackage{times}
\usepackage{psfrag,epsfig}
\usepackage{verbatim}
\usepackage{tabularx}
\usepackage[pagebackref=true,breaklinks=true,letterpaper=true,colorlinks,bookmarks=false]{hyperref}
\usepackage{cite}
\usepackage{algorithm}
\usepackage{multirow}
\usepackage{caption}
\usepackage{algorithmic}
%\usepackage[amsmath,thmmarks]{ntheorem}
\usepackage{listings}
\usepackage{color}
\usepackage{bm}



\newtheorem{thm}{Theorem}
\newtheorem{mydef}{Definition}

\DeclareMathOperator*{\rank}{rank}
\DeclareMathOperator*{\trace}{trace}
\DeclareMathOperator*{\acos}{acos}
\DeclareMathOperator*{\argmax}{argmax}


\renewcommand{\algorithmicrequire}{ \textbf{Input:}}
\renewcommand{\algorithmicensure}{ \textbf{Output:}}
\renewcommand{\mathbf}{\boldsymbol}
\newcommand{\mb}{\mathbf}
\newcommand{\matlab}[1]{\texttt{#1}}
\newcommand{\setname}[1]{\textsl{#1}}  
\newcommand{\Ce}{\mathbb{C}}
\newcommand{\Ee}{\mathbb{E}}
\newcommand{\Ne}{\mathbb{N}}
\newcommand{\Se}{\mathbb{S}}
\newcommand{\norm}[2]{\left\| #1 \right\|_{#2}}

\newenvironment{mfunction}[1]{
	\noindent
	\tabularx{\linewidth}{>{\ttfamily}rX}
	\hline
	\multicolumn{2}{l}{\textbf{Function \matlab{#1}}}\\
	\hline
}{\\\endtabularx}

\newcommand{\parameters}{\multicolumn{2}{l}{\textbf{Parameters}}\\}

\newcommand{\fdescription}[1]{\multicolumn{2}{p{0.96\linewidth}}{

		\textbf{Description}

		#1}\\\hline}

\newcommand{\retvalues}{\multicolumn{2}{l}{\textbf{Returned values}}\\}
\def\0{\boldsymbol{0}}
\def\b{\boldsymbol{b}}
\def\bmu{\boldsymbol{\mu}}
\def\e{\boldsymbol{e}}
\def\u{\boldsymbol{u}}
\def\x{\boldsymbol{x}}
\def\v{\boldsymbol{v}}
\def\w{\boldsymbol{w}}
\def\N{\boldsymbol{N}}
\def\X{\boldsymbol{X}}
\def\Y{\boldsymbol{Y}}
\def\A{\boldsymbol{A}}
\def\B{\boldsymbol{B}}
\def\y{\boldsymbol{y}}
\def\cX{\mathcal{X}}
\def\transpose{\top} % Vector and Matrix Transpose

%\long\def\answer#1{{\bf ANSWER:} #1}
\long\def\answer#1{}
\newcommand{\myhat}{\widehat}
\long\def\comment#1{}
\newcommand{\eg}{{e.g.,~}}
\newcommand{\ea}{{et al.~}}
\newcommand{\ie}{{i.e.,~}}

\newcommand{\db}{{\boldsymbol{d}}}
\renewcommand{\Re}{{\mathbb{R}}}
\newcommand{\Pe}{{\mathbb{P}}}

\hyphenation{MATLAB}

\usepackage[margin=1in]{geometry}

\begin{document}

\title{	Numerical Optimization, 2023 Fall\\Homework 2}
\date{Due 23:59 (CST), Nov. 2, 2023 }

\author{
    Name: \textbf{Zhou Shouchen} \\
	Student ID: 2021533042
}

\maketitle

\newpage
%===============================

\section{Standard Form}
Convert the following problem to a linear program in standard form. \textcolor{red}{[20pts]}~
\begin{equation}
	\begin{aligned}
		\max_{\bm{x} \in \mathbb{R}^{4}}\qquad & 2x_{1} - x_{3} + x_{4} \\
		\mathrm{s.t.}\qquad & x_{1} + x_{2} \geq 5 \\
							& x_{1} - x_{3} \leq 2 \\
                                & 4x_{2} + 3x_{3} - x_{4} \leq 10 \\
                                & x_{1} \geq 0 \\
	\end{aligned}
\end{equation}

Let $s_1,s_2,s_3$ be the slack variables for the first, second and third constraints, respectively.\\
And $s_1,s_2,s_3\geq 0$.\\
So the inequality constraints can be written as:\\
\begin{equation}
	\begin{aligned}
		 \qquad & x_{1} + x_{2} = 5 + s_1 \\
				& x_{1} - x_{3} = 2 - s_2 \\
				& 4x_{2} + 3x_{3} - x_{4} = 10 - s_3 \\
	\end{aligned}
\end{equation}

Also, the standard form should have the objective function as a minimization problem.\\
So the objective function can be written as:\\
\begin{equation}
	\begin{aligned}
		\min_{\bm{x} \in \mathbb{R}^{4}}\qquad & -(2x_{1} - x_{3} + x_{4}) \\
		i.e.
		\min_{\bm{x} \in \mathbb{R}^{4}}\qquad & -2x_{1} + x_{3} - x_{4} \\
	\end{aligned}
\end{equation}

Since there are no constraints on the boundary of $x_2$, $x_3$ and $x_4$ separately.\\
So let $x_2 = u_2 - v_2$, $x_3 = u_3 - v_3$, $x_4 = u_4 - v_4$, where $u_2,u_3,u_4,v_2,v_3,v_4\geq 0$.\\
And put them into the origin problem, we can get the standard form of the origin problem:\\

So the standard form of the origin problem is:\\
\begin{equation}
	\begin{aligned}
		\max_{x_1,u_2,u_3,u_4,v_2,v_3,v_4,s_1,s_2,s_3}\qquad & 2x_1 - u_3 + v_3 + u_4 - v_4 \\
		\mathrm{s.t.}\qquad & x_{1} + u_2 - v_2 - s_1 = 5 \\
							& x_{1} - u_3 + v_3 + s_2 = 2 \\
							& 4u_2 - 4v_2 + 3u_3 - 3v_3 - u_4 + v_4 + s_3 = 10 \\
							& x_1,u_2,u_3,u_4,v_2,v_3,v_4,s_1,s_2,s_3 \geq 0 \\
	\end{aligned}
\end{equation}

\newpage
% %===============================

\section{Two-Phase Simplex}
Use the two-phase simplex procedure to solve the following problem. \textcolor{red}{[40pts]}
\begin{equation}
	\begin{aligned}
		\min_{\bm{x} \in \mathbb{R}^{4}}\qquad & -3x_{1} + x_{2} + 3x_{3} - x_{4} \\
		\mathrm{s.t.}\qquad & x_{1} + 2x_{2} - x_{3} + x_{4} = 0 \\
							  & 2x_{1} - 2x_{2} + 3x_{3} + 3x_{4} = 9 \\
							  & x_{1} - x_{2} + 2x_{3} - x_{4} = 6 \\
							  & x_{1}, x_{2}, x_{3}, x_{4} \geq 0 \\
	\end{aligned}
\end{equation}

Since the origin problem is already the standard form, we can directly use the two-phase simplex procedure to solve it.\\
1. Phase one:\\
The supporting problem is:\\
\begin{equation}
	\begin{aligned}
		\min{\bm{x} \in \mathbb{R}^{7}}\qquad & x_5 + x_6 + x_7 \\
		\mathrm{s.t.}\qquad & x_{1} + 2x_{2} - x_{3} + x_{4} + x_5 = 0 \\
							& 2x_{1} - 2x_{2} + 3x_{3} + 3x_{4} + x_6 = 9 \\
							& x_{1} - x_{2} + 2x_{3} - x_{4} + x_7 = 6 \\
							& x_{1}, x_{2}, x_{3}, x_{4}, x_5, x_6, x_7 \geq 0 \\
	\end{aligned}
\end{equation}

And the supporting problem's simplex tableau is:\\
\begin{equation}
	\begin{aligned}
		\begin{array}{c|ccccccc|c}
			& x_1 & x_2 & x_3 & x_4 & x_5 & x_6 & x_7 & \text{b} \\
			\hline
			& 1 & 2 & -1 & 1 & 1 & 0 & 0 & 0 \\
			& 2 & -2 & 3 & 3 & 0 & 1 & 0 & 9 \\
			& 1 & -1 & 2 & -1 & 0 & 0 & 1 & 6 \\
			\hline
			c^T/r^T & 0 & 0 & 0 & 0 & 1 & 1 & 1 & 0 \\
		\end{array}
	\end{aligned}
\end{equation}

The basic is $B=(x_5,x_6,x_7)$, and $\mathbf{x}=(0,0,0,0,0,9,6)^T$.\\
Then add the row 1,2,3 to the row 4, to let the base variables' reduced cost become $0$, we can get:\\
\begin{equation}
	\begin{aligned}
		\begin{array}{c|ccccccc|c}
			& x_1 & x_2 & x_3 & x_4 & x_5 & x_6 & x_7 & \text{b} \\
			\hline
			& \boxed{1} & 2 & -1 & 1 & 1 & 0 & 0 & 0 \\
			& 2 & -2 & 3 & 3 & 0 & 1 & 0 & 9 \\
			& 1 & -1 & 2 & -1 & 0 & 0 & 1 & 6 \\
			\hline
			r^T & -4 & 1 & -4 & -3 & 0 & 0 & 0 & -15 \\
		\end{array}
	\end{aligned}
\end{equation}

The basic is $B=(x_5,x_6,x_7)$.\\
We choose the leftmost column with negative reduced cost, which is $x_1$.\\
And we choose the row with the minimum ratio, which is row 1, and pivot, let $x_1$ in base and $x_5$ out base.\\

\begin{equation}
	\begin{aligned}
		\begin{array}{c|ccccccc|c}
			& x_1 & x_2 & x_3 & x_4 & x_5 & x_6 & x_7 & \text{b} \\
			\hline
			& 1 & 2 & -1 & 1 & 1 & 0 & 0 & 0 \\
			& 0 & -6 & \boxed{5} & 1 & -2 & 1 & 0 & 9 \\
			& 0 & -3 & 3 & -2 & -1 & 0 & 1 & 6 \\
			\hline
			r^T & 0 & 9 & -8 & 1 & 4 & 0 & 0 & -15 \\
		\end{array}
	\end{aligned}
\end{equation}

The basic is $B=(x_1,x_6,x_7)$.\\
We choose the leftmost column with negative reduced cost, which is $x_3$.\\
And we choose the row with the minimum ratio, which is row 2, and pivot, let $x_3$ in base and $x_6$ out base.\\

\begin{equation}
	\begin{aligned}
		\begin{array}{c|ccccccc|c}
			& x_1 & x_2 & x_3 & x_4 & x_5 & x_6 & x_7 & \text{b} \\
			\hline
			& 1 & \frac{4}{5} & 0 & \frac{6}{5} & \frac{3}{5} & \frac{1}{5} & 0 & \frac{9}{5} \\
			& 0 & -\frac{6}{5} & 1 & \frac{1}{5} & -\frac{2}{5} & \frac{1}{5} & 0 & \frac{9}{5} \\
			& 0 & \boxed{\frac{3}{5}} & 0 & -\frac{13}{5} & -\frac{1}{5} & -\frac{3}{5} & 1 & \frac{3}{5} \\
			\hline
			r^T & 0 & -\frac{3}{5} & 0 & \frac{13}{5} & \frac{4}{5} & \frac{8}{5} & 0 & -\frac{3}{5} \\
		\end{array}
	\end{aligned}
\end{equation}

The basic is $B=(x_1,x_3,x_7)$.\\
We choose the leftmost column with negative reduced cost, which is $x_2$.\\
And we choose the row with the minimum ratio, which is row 3, and pivot, let $x_2$ in base and $x_7$ out base.\\

\begin{equation}
	\begin{aligned}
		\begin{array}{c|ccccccc|c}
			& x_1 & x_2 & x_3 & x_4 & x_5 & x_6 & x_7 & \text{b} \\
			\hline
			& 1 & 0 & 0 & \frac{14}{3} & \frac{1}{3} & 1 & -\frac{4}{3} & 1 \\
			& 0 & 0 & 1 & -5 & 0 & -1 & 2 & 3 \\
			& 0 & 1 & 0 & -\frac{13}{3} & \frac{1}{3} & -1 & \frac{5}{3} & 1 \\
			\hline
			r^T & 0 & 0 & 0 & 0 & 1 & 1 & 1 & 0 \\
		\end{array}
	\end{aligned}
\end{equation}

The basic is $B=(x_1,x_2,x_3)$.\\
And all the reduced cost are non-negative, so the supporting problem is feasible.\\
So the phase one is finished.\\
And the basic feasible solution is $\mathbf{x}=(1,1,3,0,0,0,0)^T$.\\

2. Phase two:\\
The tableau of the origin problem is:\\
\begin{equation}
	\begin{aligned}
		\begin{array}{c|ccccccc|c}
			& x_1 & x_2 & x_3 & x_4 & \text{b} \\
			\hline
			& 1 & 0 & 0 & \frac{14}{3} & 1 \\
			& 0 & 0 & 1 & -5 & 3 \\
			& 0 & 1 & 0 & -\frac{13}{3} & 1 \\
			\hline
			c^T/r^T & -3 & 1 & 3 & -1 & 0 \\
		\end{array}
	\end{aligned}
\end{equation}

Then let the base variables' reduced cost become $0$, we can get:\\
\begin{equation}
	\begin{aligned}
		\begin{array}{c|ccccccc|c}
			& x_1 & x_2 & x_3 & x_4 & \text{b} \\
			\hline
			& 1 & 0 & 0 & \frac{14}{3} & 1 \\
			& 0 & 0 & 1 & -5 & 3 \\
			& 0 & 1 & 0 & -\frac{13}{3} & 1 \\
			\hline
			r^T & 0 & 0 & 0 & \frac{97}{3} & -7 \\
		\end{array}
	\end{aligned}
\end{equation}

So above all, the basic feasible solution of the origin problem is $\mathbf{x}=(1,1,3,0)^T$.\\
And the optimal value is $7$.\\

\newpage
% %===============================

\section{Extreme Point}
\subsection{Q1}
Prove that the extreme points of the following two sets are in one-to-one correspondence.
\textcolor{red}{[20pts]}
\begin{equation}
	\begin{aligned}
		& S_{1} = \{ \bm{x} \in \mathbb{R}^{n} : \bm{Ax} \leq \bm{b}, \bm{x} \geq 0 \} \\
		& S_{2} = \{ \bm{(x, y)} \in \mathbb{R}^{n} \times \mathbb{R}^{m} : \bm{Ax} + \bm{y} = \bm{b}, \bm{x} \geq 0, \bm{y} \geq 0 \} \\
	\end{aligned}
\end{equation}
 
, where $\bm{A} \in \mathbb{R}^{m \times n}$, $\bm{b} \in \mathbb{R}^{m}$.


Suppose that the extreme points of $S_1$ compose the set $P_1$.\\
And the extreme points of $S_2$ compose the set $P_2$.\\

$\forall\mathbf{x}\in P_1$





So $P_1\subseteq P_2$.\\

Similarly, $\forall(\mathbf{x},\mathbf{y})\in P_2$





So $P_2\subseteq P_1$.\\

Since $P_1\subseteq P_2$ and $P_2\subseteq P_1$
So the extreme points of the sets $S_1,S_2$ are one-to-one correspondence.\\

\newpage
\subsection{Q2}
Does the set $P = \{ \bm{x} \in \mathbb{R}^{2} : 0 \leq x_{1} \leq 1 \}$ have extreme points? What is its standard form? Does it have extreme points in its standard form? If so, give a extreme point and explain why it is a extreme point.
\textcolor{red}{[20pts]}







So above all, $P$ has no extreme points.\\
The standard form of $P$ is:\\


The standard form has extreme points, and $...$ is one of the extreme points. The reasons are above.\\

\end{document}


