\documentclass[10pt]{article}
\usepackage[UTF8]{ctex}

\usepackage[utf8]{inputenc} % allow utf-8 input
\usepackage{amsthm}
\usepackage{amsmath,amscd}
\usepackage{amssymb,array}
\usepackage{amsfonts,latexsym}
\usepackage{graphicx,subfig,wrapfig}
\usepackage{times}
\usepackage{psfrag,epsfig}
\usepackage{verbatim}
\usepackage{tabularx}
\usepackage[pagebackref=true,breaklinks=true,letterpaper=true,colorlinks,bookmarks=false]{hyperref}
\usepackage{cite}
\usepackage{algorithm}
\usepackage{multirow}
\usepackage{caption}
\usepackage{algorithmic}
%\usepackage[amsmath,thmmarks]{ntheorem}
\usepackage{listings}
\usepackage{color}
\usepackage{bm}



\newtheorem{thm}{Theorem}
\newtheorem{mydef}{Definition}

\DeclareMathOperator*{\rank}{rank}
\DeclareMathOperator*{\trace}{trace}
\DeclareMathOperator*{\acos}{acos}
\DeclareMathOperator*{\argmax}{argmax}


\renewcommand{\algorithmicrequire}{ \textbf{Input:}}
\renewcommand{\algorithmicensure}{ \textbf{Output:}}
\renewcommand{\mathbf}{\boldsymbol}
\newcommand{\mb}{\mathbf}
\newcommand{\matlab}[1]{\texttt{#1}}
\newcommand{\setname}[1]{\textsl{#1}}  
\newcommand{\Ce}{\mathbb{C}}
\newcommand{\Ee}{\mathbb{E}}
\newcommand{\Ne}{\mathbb{N}}
\newcommand{\Se}{\mathbb{S}}
\newcommand{\norm}[2]{\left\| #1 \right\|_{#2}}

\newenvironment{mfunction}[1]{
	\noindent
	\tabularx{\linewidth}{>{\ttfamily}rX}
	\hline
	\multicolumn{2}{l}{\textbf{Function \matlab{#1}}}\\
	\hline
}{\\\endtabularx}

\newcommand{\parameters}{\multicolumn{2}{l}{\textbf{Parameters}}\\}

\newcommand{\fdescription}[1]{\multicolumn{2}{p{0.96\linewidth}}{

		\textbf{Description}

		#1}\\\hline}

\newcommand{\retvalues}{\multicolumn{2}{l}{\textbf{Returned values}}\\}
\def\0{\boldsymbol{0}}
\def\b{\boldsymbol{b}}
\def\bmu{\boldsymbol{\mu}}
\def\e{\boldsymbol{e}}
\def\u{\boldsymbol{u}}
\def\x{\boldsymbol{x}}
\def\v{\boldsymbol{v}}
\def\w{\boldsymbol{w}}
\def\N{\boldsymbol{N}}
\def\X{\boldsymbol{X}}
\def\Y{\boldsymbol{Y}}
\def\A{\boldsymbol{A}}
\def\B{\boldsymbol{B}}
\def\y{\boldsymbol{y}}
\def\cX{\mathcal{X}}
\def\transpose{\top} % Vector and Matrix Transpose

%\long\def\answer#1{{\bf ANSWER:} #1}
\long\def\answer#1{}
\newcommand{\myhat}{\widehat}
\long\def\comment#1{}
\newcommand{\eg}{{e.g.,~}}
\newcommand{\ea}{{et al.~}}
\newcommand{\ie}{{i.e.,~}}

\newcommand{\db}{{\boldsymbol{d}}}
\renewcommand{\Re}{{\mathbb{R}}}
\newcommand{\Pe}{{\mathbb{P}}}

\hyphenation{MATLAB}

\usepackage[margin=1in]{geometry}

\begin{document}

\title{	Numerical Optimization, 2023 Fall\\Homework 2}
\date{Due 23:59 (CST), Nov. 2, 2023 }

\author{
    Name: \textbf{Zhou Shouchen} \\
	Student ID: 2021533042
}

\maketitle

\newpage
%===============================

\section{Standard Form}
Convert the following problem to a linear program in standard form. \textcolor{red}{[20pts]}~
\begin{equation}
	\begin{aligned}
		\max_{\bm{x} \in \mathbb{R}^{4}}\qquad & 2x_{1} - x_{3} + x_{4} \\
		\mathrm{s.t.}\qquad & x_{1} + x_{2} \geq 5 \\
							& x_{1} - x_{3} \leq 2 \\
                                & 4x_{2} + 3x_{3} - x_{4} \leq 10 \\
                                & x_{1} \geq 0 \\
	\end{aligned}
\end{equation}

\newpage
% %===============================

\section{Two-Phase Simplex}
Use the two-phase simplex procedure to solve the following problem. \textcolor{red}{[40pts]}
\begin{equation}
	\begin{aligned}
		\min_{\bm{x} \in \mathbb{R}^{4}}\qquad & -3x_{1} + x_{2} + 3x_{3} - x_{4} \\
		\mathrm{s.t.}\qquad & x_{1} + 2x_{2} - x_{3} + x_{4} = 0 \\
							  & 2x_{1} - 2x_{2} + 3x_{3} + 3x_{4} = 9 \\
							  & x_{1} - x_{2} + 2x_{3} - x_{4} = 6 \\
							  & x_{1}, x_{2}, x_{3}, x_{4} \geq 0 \\
	\end{aligned}
\end{equation}

\newpage
% %===============================

\section{Extreme Point}
\subsection{Q1}
Prove that the extreme points of the following two sets are in one-to-one correspondence.
\textcolor{red}{[20pts]}
\begin{equation}
	\begin{aligned}
		& S_{1} = \{ \bm{x} \in \mathbb{R}^{n} : \bm{Ax} \leq \bm{b}, \bm{x} \geq 0 \} \\
		& S_{2} = \{ \bm{(x, y)} \in \mathbb{R}^{n} \times \mathbb{R}^{m} : \bm{Ax} + \bm{y} = \bm{b}, \bm{x} \geq 0, \bm{y} \geq 0 \} \\
	\end{aligned}
\end{equation}
 
, where $\bm{A} \in \mathbb{R}^{m \times n}$, $\bm{b} \in \mathbb{R}^{m}$.

~\\
~\\
~\\
~\\
~\\
~\\

\newpage
\subsection{Q2}
Does the set $P = \{ \bm{x} \in \mathbb{R}^{2} : 0 \leq x_{1} \leq 1 \}$ have extreme points? What is its standard form? Does it have extreme points in its standard form? If so, give a extreme point and explain why it is a extreme point.
\textcolor{red}{[20pts]}

\end{document}


